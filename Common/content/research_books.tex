

%----------------------------------------------------------------------------------------
%	Books
%----------------------------------------------------------------------------------------
%\pagebreak
\begin{rSection}{\textrm{Open-source Books and Book Chapters}}%:
\vspace{-1mm}\begin{center}\footnotesize{denotes WFU *undergraduate* or $\dagger$graduate$\dagger$ student co-author; (First Published–Latest Update+)}\end{center}\vspace{-1mm}
{\large {\bf Open Source Books}}\begin{etaremune}
\item  \meb (2021–2025+). Data Science for Psychologists. \textit{Free Multimedia Textbook Written in \R about \R}. \bluelink{https://datascience4psych.github.io/DataScience4Psych/}{Read it here: datascience4psych.github.io/DataScience4Psych/}
\end{etaremune}
{\large \bf Book Chapters}
\begin{etaremune}
\item \meb, Tyson, H. K.*, \& Lyu, X$\dagger$ (in press). Pragmatic AI Integration in R Teaching: From Frustration to Facilitation.  In I. Katzarska-Miller, M. Jackson, \& M. Fortner,  (Eds.) (2026) \textit{Integrating Artificial Intelligence into the Psychology Classroom} Retrieved from the Society for the Teaching of Psychology: \href{https://teachpsych.org/ebooks/index}{\small\color{blue}{teachpsych.org/ebooks/}}
\end{etaremune}
%\pagebreak
\end{rSection}%\vspace{-4mm}
